%Software Engineering design requirements
\documentclass{article}
%\addtolength{\textwidth}{0.6in}
%\addtolength{\hoffset}{-0.3in}
%\addtolength{\textheight}{cin}
%\addtolength{\voffset}{-cin}

\begin{document}
\title{Software Engineering Design Requirements}
\author{
Sandro Badame\\
Anthony DiFiore\\
Christopher Eichel\\
Andrew Esca\\
Andres Ramirez
}
\date{\today}
\maketitle
This is the proposed design for a desktop application that will generate an decision tree using data parsed from an excel spreadsheet.
The application will be programed using the Java language along with a collection of libraries will each help provide a needed feature.

The UI will be created using the Java Swing Libraries; most of the code for the UI will be generated by the Netbeans Swing editor. Since Swing is very much an event driven API the code of the application will follow suite. Every ``UI'' event triggers a function call is in a class. The UI consists of five buttons that go horizontally along the top, two sliders vertically stacked below the buttons and the actual decision tree below the sliders. Each button does one of the following: Load an excel spreadsheet, email the graph to the user's address, save the graph to the harddrive, print the graph and load a help screen. The sliders will be used to modify the paramters of the tree in realtime. At first most of the UI, except the load button will be disabled. Once an excel spreadsheet has been loaded then the UI becomes enabled. 

The design will be implemented using a top down approach that revolves around a singleton Graph class that holds the current state of the program. 
This includes the answer key, student answers, decision tree, goodness and noise numbers. 
The Graph class uses the ExcelReader class to parse information from the source excel file, from that it then runs 'boothe algorithm' to generate a tree. 
Once a decision tree is generated the the UI is then enabled and allowed to make modifications to the tree.
The POI-HSSF library will be used to parse the excel spreadsheets and retreive the needed information to build the decision tree.
The GRender class and the UI are then able to use this tree instance to display and modify the tree.
The EmailSender class will use the JavaMail library to send an e-mail. It will be researched if an email account for this app can be obtained.
The GRrender class will use the JGraph library to create and display a graph to the user. 

\end{document}
